\documentclass[a4paper,12pt]{article}
\usepackage[margin=0cm]{geometry}
\usepackage{fontspec}
\usepackage{xcolor}
\usepackage{graphicx}
\usepackage{enumitem}
\usepackage{paracol}
\usepackage{multicol}
\usepackage[hidelinks]{hyperref}
\usepackage{titlesec}
\usepackage{hyperxmp}
\usepackage[most]{tcolorbox}
\usepackage{parskip} % Pour une belle séparation
\usepackage{enumitem}
\usepackage{tabularx}
\usepackage[normalem]{ulem} % Remplace \underline par \uline avec options

\hypersetup{
  pdftitle={CV - Simon Gallien},
  pdfauthor={Simon Gallien},
  pdfsubject={Curriculum Vitae},
  pdfkeywords={Data Science, Python, Machine Learning, CV},
  pdfproducer={XeLaTeX},
  pdfcreator={XeLaTeX},
  pdfa=true,
  colorlinks=false,
  pdfapart=3,
  pdfaconformance=B
}

\usepackage{tikz}
\newcommand{\tagskill}[1]{%
  \tikz[baseline]{
    \node[inner sep=0pt, outer sep=0pt, anchor=base] (X) {\strut #1};
    \draw[black!50, line width=0.2pt] 
      ([yshift=-0.3ex]X.south west) -- ([yshift=-0.3ex]X.south east);
  }%
}


% bulle langues
\usepackage{tikz}

\newcommand{\languagerow}[3]{
  \noindent
  \textbf{#1}
  \hspace{0.5em}
  \begin{tikzpicture}[baseline={(0,-0.12)}]
    \foreach \x in {1,...,5} {
      \ifnum\x>#3
        \filldraw[fill=gray!30, draw=gray!30] (\x*0.5,0) circle (0.45em);
      \else
        \filldraw[fill=bleufonce, draw=bleufonce] (\x*0.5,0) circle (0.45em);
      \fi
    }
  \end{tikzpicture}\\[0.2em]
  {\small #2} \\[0.6em]
}


\definecolor{bleufonce}{RGB}{0, 51, 102} % à mettre dans le préambule

\titlespacing*{\section}{0pt}{1em}{0.4em}

\titleformat{\section}
  % {\bfseries\large\MakeUppercase}{}{0pt}{}
  {\color{bleufonce}\bfseries\large\MakeUppercase}
  {}{0pt}{}

\newcommand{\sectionrule}[1]{%
  \section*{#1}%
  \vspace{-1em}%
  {\color{bleufonce}\noindent\rule{\linewidth}{1pt}}%
  \vspace{0.5em}%
}


% \definecolor{bgleft}{gray}{0.8}
% \definecolor{bgright}{gray}{0.8}

% === Polices personnalisées ===
\setmainfont{TeX Gyre Heros}
\newfontfamily\titlefont{TeX Gyre Heros}
\setmonofont{Courier New} 



\pagestyle{empty}

% === Pas d'indentation
\setlength{\parindent}{0pt}
\setlength{\parskip}{0.5em}

\begin{document}
% === En-tête pleine largeur ===

\vspace*{1em}
\noindent
\hspace*{2.5em}
\begin{minipage}[t]{0.02\linewidth}
  \colorbox{bleufonce}{\color{bleufonce}\rule{2pt}{2em}}\\[-1pt]
  \colorbox{bleufonce}{\color{bleufonce}\rule{2pt}{2em}}\\[1pt]
\end{minipage}
\hspace{1em}
\begin{minipage}[t]{\dimexpr\linewidth-3em\relax}
  {\titlefont\Huge \textcolor{bleufonce}{\textbf{SIMON GALLIEN}}}\\[1em]
  {\titlefont\Large {Alternant Data Scientist}}

\end{minipage}
% \vspace{2em}

\vspace{-0.5em}

% Choisir les largeurs
\setcolumnwidth{.3\textwidth, .7\textwidth}

\begin{paracol}{2}

% === Colonne de gauche ===
\begin{leftcolumn}
% \colorbox{bgleft}{%
  \begin{minipage}[t][\dimexpr\textheight - 9em\relax][t]{\dimexpr\linewidth}
    \hspace*{2.5em}
    \parbox{\dimexpr\linewidth-1.5em\relax}{

    \section*{Contact}
    \small
    +33 6 31 02 11 55\\
    \href{mailto:simongallien@orange.fr}{simongallien@orange.fr}\\
    30100, Alès\\
    Mobilité : France\\
    % simongallien.com\\
    linkedin.com/in/simongallien
    github.com/SimonGallien

    \vspace{1em}
    \section*{Compétences} 
    % Dans la section compétences :
    \noindent
    \textbf{Langages \& bibliothèques :}\\[0.5em]
    \tagskill{Python} \hspace{0.5em} \tagskill{Pandas} \hspace{0.5em} \tagskill{Scikit-learn} \\[0.5em]
    \tagskill{NumPy} \hspace{0.5em} \tagskill{Seaborn} \hspace{0.5em} \tagskill{Matplotlib}
   
    \vspace{1.5em}
    \noindent
    \textbf{Systèmes \& outils :}\\[0.5em]
    \tagskill{Windows (WSL)} \hspace{0.5em} \tagskill{Linux}\\[0.5em]
    \tagskill{Git} \hspace{0.5em} \tagskill{GitHub} \hspace{0.5em} \tagskill{Docker} \\[0.5em]
    \tagskill{Poetry} \hspace{0.5em} \tagskill{Pyenv}

    \vspace{1.5em}
    \noindent
    \textbf{Bases de données :}\\[0.5em]
    \tagskill{MySQL} \hspace{0.5em} \tagskill{MongoDB}

    \vspace{1.5em}
    \noindent
    \textbf{Statistiques :}\\[0.5em]
    \tagskill{Analyse exploratoire} \\[0.5em]
    \tagskill{Tests d’hypothèses} \\  

    \vspace{0.5em}
    \section*{Langues}
    \languagerow{Français \hspace{0.5em}}{Langue maternelle}{5}
    \languagerow{Anglais \hspace{1em}}{Compétent (C1)}{3}
    \languagerow{Espagnol \hspace{0.1em}}{Intermédiaire (B1)}{2}


    \vspace{-0.7em}
    \section*{Soft Skills}
    \begin{itemize}[leftmargin=1em, nosep]
      \item Rigueur
      \item Autonomie
      \item Curiosité
      \item Esprit critique
    \end{itemize}

    % \vspace{1em}
    % \section*{Intérêts}
    % Défense, cybersécurité, sciences

    } % fin parbox
  \end{minipage}
% }
\end{leftcolumn}

% === Colonne de droite ===
\begin{rightcolumn}
% \colorbox{bgright}{%
\begin{minipage}[t][\dimexpr\textheight - 9em\relax][t]{\dimexpr\linewidth - 1.5em\relax}
  \hspace*{1.5em}
  \parbox{\dimexpr\linewidth - 1.5em\relax}{

  \section*{Profil}
  Ingénieur en reconversion vers la Data Science, je me forme avec rigueur et autonomie.
  Je cherche une alternance pour mettre en pratique mes compétences en machine learning ainsi que mon esprit d’analyse.

  \vspace{0.7em}

\sectionrule{Formation}
% \vspace{-3em}
\textbf{Data Science} – Machine Learnia \hfill {\color{blue}2025 (en cours)}
\begin{itemize}[leftmargin=1em, nosep]
  \vspace{0.2em}
  \item Formation intensive en Data Science : Python, Machine Learning, Deep Learning, SQL, Mathématiques appliquées
  \vspace{0.2em}
  \item Projets pratiques : Kaggle, traitement de données réelles, modélisation avancée
\end{itemize}
% \vspace{0.5em}

\vspace{0.5em}
\textbf{Diplôme d'ingénieur} – ESTIA \hfill {\color{blue}2013–2016}
\begin{itemize}[leftmargin=1em, nosep]
  \vspace{0.2em}
  \item Formation généraliste d’ingénieur : mathématiques, physique, informatique, gestion de projet
  \vspace{0.2em}
  \item Développement de compétences analytiques et techniques solides
\end{itemize}

\vspace{0.7em}
\sectionrule{Expériences Professionnelles}

\textbf{Ingénieur Data et exploitation – Team CMR} \hfill {\color{blue}2017–2024}  
\begin{itemize}[leftmargin=1em, nosep]
  \vspace{0.2em}
  \item Mis en place des \textbf{outils de collecte et de traitement de données} pour le suivi des performances.
  \vspace{0.2em}
  \item Réalisé des \textbf{analyses de données} pour la \textbf{prise de décision technique et stratégique}.
  \vspace{0.2em}
  \item Collaboré avec les \textbf{équipes techniques et les pilotes} dans un environnement exigeant.
  \vspace{0.2em}
  \item Responsable de programmes sur plusieurs championnats (\textbf{Alpine Cup}, \textbf{GT4 France \& Europe}) avec une contribution directe à \textbf{plus de 30 podiums} en 6 saisons.
\end{itemize}

\vspace{0.7em}
\sectionrule{Projets de Machine Learning}

\textbf{Compétition Kaggle – Machine Learnia ($\approx$ 500 élèves)} \hfill {\color{blue}2025}  
\begin{itemize}[leftmargin=1em, nosep]
  \vspace{0.2em}
  \item Modèle de classification avec \textbf{Random Forest} (Scikit-learn) pour prédire les départs clients.
  \item Objectif : maximiser le \textbf{F1-score}. 
  \item Résultat : \textbf{F1 = 0.66091}, classé \textbf{14ème} à \textbf{0.00392 pts} du top 3.
\end{itemize}

\vspace{0.7em}

\textbf{Compétition Kaggle – Machine Learnia ($\approx$ 500 élèves)} \hfill {\color{blue}2025}  
\begin{itemize}[leftmargin=1em, nosep]
  \vspace{0.2em}
  \item Modèle de régression avec \textbf{XGBoost} pour estimer le prix de vente d’un bien immobilier.
  \item Objectif : minimiser la \textbf{MAE}. 
  \item Résultat : \textbf{MAE = 17195.09}, classé \textbf{5ème} à \textbf{348 \$} du top 3.
\end{itemize}

} % fin parbox
\end{minipage}
% }
\end{rightcolumn}

\end{paracol}
\end{document}

\documentclass[a4paper,12pt]{article}
\usepackage[margin=0cm]{geometry}
\usepackage{fontspec}
\usepackage{xcolor}
\usepackage{graphicx}
\usepackage{enumitem}
\usepackage{paracol}
\usepackage{multicol}
\usepackage[hidelinks]{hyperref}
\usepackage{titlesec}
\titlespacing*{\section}{0pt}{1em}{0.4em}

\titleformat{\section}
  {\bfseries\large\MakeUppercase}{}{0pt}{}

\newcommand{\sectionrule}[1]{%
\section*{#1}%
\vspace{-1em}%
\noindent\rule{\linewidth}{1pt}%
\vspace{0.5em}%
}

\definecolor{bgleft}{gray}{0.8}
\definecolor{bgright}{gray}{0.85}

% === Polices personnalisées ===
\setmainfont{TeX Gyre Heros}
\newfontfamily\titlefont{TeX Gyre Heros}
\setmonofont{Courier New}             

\pagestyle{empty}

% === Pas d'indentation
\setlength{\parindent}{0pt}
\setlength{\parskip}{0.5em}

\begin{document}
% === En-tête pleine largeur ===
% \noindent
% \makebox[\paperwidth]{%
%   \parbox{\paperwidth}{%
%     \vspace{2.5em}
%     \hspace*{1.5em}
%     {\titlefont\Huge \textbf{SIMON GALLIEN}}\\[1em]
%     \hspace*{1.5em}
%     {\titlefont\Huge \textsc{Alternant Data Scientist}}
%     \vspace{2em}
%   }%
% }

\vspace*{0.5em}
\noindent
\hspace*{1.5em}
\begin{minipage}[t]{0.02\linewidth}
  \colorbox{black}{\rule{2pt}{2em}}\\[-1pt]
  \colorbox{black}{\rule{2pt}{2em}}\\[1pt]
\end{minipage}
\hspace{1em}
\begin{minipage}[t]{\dimexpr\linewidth-3em\relax}
  {\titlefont\Huge \textbf{SIMON GALLIEN}}\\[1em]
  {\titlefont\Large \textsc{Alternant Data Scientist}}
\end{minipage}
% \vspace{2em}




\vspace{-0.5em}

% Choisir les largeurs
\setcolumnwidth{.25\textwidth, .75\textwidth}

\begin{paracol}{2}

% === Colonne de gauche ===
\begin{leftcolumn}
% \colorbox{bgleft}{%
  \begin{minipage}[t][\dimexpr\textheight - 9em\relax][t]{\dimexpr\linewidth}
    \hspace*{1.5em}
    \parbox{\dimexpr\linewidth-1.5em\relax}{

    \section*{Contact}
    \small
    +33 6 31 02 11 55\\
    \href{mailto:simongallien@orange.fr}{simongallien@orange.fr}\\
    30100, Alès\\
    Mobilité : France\\
    % simongallien.com\\
    linkedin.com/in/simongallien


    \vspace{1em}
    \section*{Formation}
    \textbf{Data Science} (en cours)\\
    \footnotesize 2025 · Machine Learnia 
    \begin{itemize}[leftmargin=1em, nosep]
      \item Machine Learning
      \item Deep Learning
      \item Python
      \item SQL
      \item Mathématiques appliquées\\
    \end{itemize}

    \textbf{Développeur Web}\\
    \footnotesize 2024 · OpenClassrooms
    \begin{itemize}[leftmargin=1em, nosep]
      \item HTML, CSS
      \item JavaScript
      \item React
      \item Node.js\\
    \end{itemize}

    \textbf{École d'ingénieur}\\
    \footnotesize 2016 · ESTIA (Bidart)
    \begin{itemize}[leftmargin=1em, nosep]
      \item Formation généraliste
    \end{itemize}

    \vspace{1em}
    \section*{Langues}
    Français : maternelle\\
    Anglais : C1\\
    Espagnol : B1

    \vspace{1em}
    \section*{Soft Skills}
    \begin{itemize}[leftmargin=1em, nosep]
      \item Rigueur
      \item Autonomie
      \item Curiosité
      \item Esprit critique
    \end{itemize}

    % \vspace{1em}
    % \section*{Intérêts}
    % Défense, cybersécurité, sciences

    } % fin parbox
  \end{minipage}
% }
\end{leftcolumn}

% === Colonne de droite ===
\begin{rightcolumn}
% \colorbox{bgright}{%
\begin{minipage}[t][\dimexpr\textheight - 9em\relax][t]{\dimexpr\linewidth - 1.5em\relax}
  \hspace*{1em}
  \parbox{\dimexpr\linewidth - 1.5em\relax}{

  \section*{Profil}
  Ingénieur en reconversion vers la Data Science, je me forme avec rigueur et autonomie.
  Je cherche une alternance pour mettre en pratique mes compétences en machine learning et mon esprit d’analyse.

  \vspace{0.5em}
  % \sectionrule{Compétences}
  % \begin{itemize}[leftmargin=1em, nosep]
  %   \item \textbf{Systèmes} : Windows (WSL compris), Linux
  %   \item \textbf{Langages} : Python, JavaScript, HTML, CSS, SQL
  %   \item \textbf{Librairies Python} : Pandas, NumPy, Scikit-learn, Matplotlib, Seaborn, XGBoost
  %   \item \textbf{Frameworks} : React, Node.js
  %   \item \textbf{Bases de données} : MySQL, PostgreSQL, MongoDB
  %   \item \textbf{Outils} : Git, Github, Docker, Poetry, Pyenv
  %   \item \textbf{Machine Learning} : Régression, classification, clustering, réseaux de neurones
  %   \item \textbf{Statistiques} : Analyse exploratoire, tests d’hypothèses
  % \end{itemize}

\sectionrule{Compétences}
\vspace{-3em}
\setlength{\columnsep}{2em}
\begin{multicols}{2}

\textbf{Systèmes}
\begin{itemize}[leftmargin=1.2em, nosep]
  \item Windows (WSL), Linux
\end{itemize}
\vspace{1em}
\textbf{Data Science}
\begin{itemize}[leftmargin=1.2em, nosep]
  \item \textbf{Python :} Pandas, NumPy, Scikit-learn, XGBoost,
   Matplotlib, Seaborn

\end{itemize}

\vspace{1em}
\textbf{Statistiques}
\begin{itemize}[leftmargin=1.2em, nosep]
  \item Analyse exploratoire
  \item Tests d’hypothèses
\end{itemize}

\vspace{1em}
\textbf{Web}
\begin{itemize}[leftmargin=1.2em, nosep]
  \item HTML, CSS, JavaScript
  \item React, Node.js
\end{itemize}

\vspace{1em}
\textbf{Bases de données}
\begin{itemize}[leftmargin=1.2em, nosep]
  \item MySQL
  \item PostgreSQL
  \item MongoDB
\end{itemize}

\vspace{1em}
\textbf{Outils}
\begin{itemize}[leftmargin=1.2em, nosep]
  \item Git, GitHub, Docker, Poetry, Pyenv
\end{itemize}

\end{multicols}
\vspace{0em}

  % \sectionrule{Expériences Professionnelles}
  % \textbf{2017 - 2024 · Team CMR : Ingénieur Data et exploitation}
  % \begin{itemize}[leftmargin=1em, nosep]
  %   \item Développement d'outils de gestion de données
  %   \item Analyse de données pour la prise de décision
  %   \item Collaboration avec les équipes techniques et métiers\\
  % \end{itemize}
  % Responsable des programmes :
  % \begin{itemize}[leftmargin=1em, nosep]
  %   \item Alpine Cup en 2018 et 2019 : Champion durant ces 2 années
  %   \item GT4 France et Europe de 2020 à 2023 : Plusieurs victoires et podiums, vice champion GT4 France 2023 avec une Alpine A110 GT4
  %   \item GT4 European Series de 2020 à 2021 : Victoire générale à Spa-Francorchamp avec Toyota et plusieurs podiums \\
  % \end{itemize}

  % \textbf{2007 - 2010 · Renault : Mécanicien/Technicien} \\
  % Stages de formation BEP puis BAC Pro + job d'été en agence Renault\\
  % J'ai appris à avoir de la rigueur, de l'autonomie, organisation de poste de travail,
  % et à comprendre les enjeux d'une entreprise.

\sectionrule{Expériences Professionnelles}

\textbf{Ingénieur Data et exploitation – Team CMR} \hfill {\color{blue}2017–2024}  
\begin{itemize}[leftmargin=1em, nosep]
  \vspace{0.2em}
  \item Mis en place des \textbf{outils de collecte et de traitement de données} pour le suivi des performances.
  \vspace{0.2em}
  \item Réalisé des \textbf{analyses de données} pour la \textbf{prise de décision technique et stratégique}.
  \vspace{0.2em}
  \item Collaboré avec les \textbf{équipes techniques et les pilotes} dans un environnement exigeant.
  \vspace{0.2em}
  \item Responsable de programmes sur plusieurs championnats (\textbf{Alpine Cup}, \textbf{GT4 France \& Europe}) avec une contribution directe à \textbf{plus 30 podiums} en 6 saisons.
\end{itemize}






  % \vspace{1em}
  % \textbf{Mécanicien/Technicien – Renault · 2007–2010}
  % \begin{itemize}[leftmargin=1em, nosep]
  %   \item Stages BEP/BAC Pro et job d’été en agence Renault.
  %   \item Rigueur, autonomie et organisation du poste de travail.
  %   \item Compréhension des enjeux techniques et humains d’un atelier automobile.
  % \end{itemize}


%   \sectionrule{Expériences Personnelles}
%  \textbf{2025 · Machine Learnia : Compétition Kaggle Bank-churn} \\
%   Modèle de classification : Random Forest de scikit-learn.\\
%   Objectif : maximiser le score F1.\\
%   Score F1 : 0.66091, classé 14ème à 0.00392 points du top 3.
%   \vspace{0.5em}\\
%   \textbf{2025 · Machine Learnia : Compétition Kaggle Prix immobilier} \\
%   Modèle de régression : XGBoost\\
%   Objectif : minimiser l'erreur moyenne absolue (MAE).\\
%   Score MAE : 17195.09186, classé 5ème à 348.0651 dollards du top 3.
\vspace{0.5em}
\sectionrule{Expériences Personnelles}

\textbf{Compétition Kaggle – Machine Learnia} \hfill {\color{blue}2025}  
\begin{itemize}[leftmargin=1em, nosep]
  \vspace{0.2em}
  \item Modèle de classification avec \textbf{Random Forest} (Scikit-learn) pour prédire les départs clients.
  \item Objectif : maximiser le \textbf{F1-score}. 
  \item Résultat : \textbf{F1 = 0.66091}, classé \textbf{14ème} à \textbf{0.00392 pts} du top 3.
\end{itemize}

\vspace{0.7em}

\textbf{Compétition Kaggle – Machine Learnia} \hfill {\color{blue}2025}  
\begin{itemize}[leftmargin=1em, nosep]
  \vspace{0.2em}
  \item Modèle de régression avec \textbf{XGBoost} pour estimer le prix de vente d’un bien immobilier.
  \item Objectif : minimiser la \textbf{MAE}. 
  \item Résultat : \textbf{MAE = 17195.09}, classé \textbf{5ème} à \textbf{348 \$} du top 3.
\end{itemize}

} % fin parbox
\end{minipage}
% }
\end{rightcolumn}

\end{paracol}
\end{document}
